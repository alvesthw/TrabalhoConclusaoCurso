\chapter{RESULTADOS ESPERADOS}

Espera-se que, ao final deste trabalho, o modelo baseado em \glspl{cnn} seja capaz de identificar, com alta acurácia, anomalias visuais em imagens de jogos digitais. A partir da aplicação de técnicas de processamento de imagens e treinamento supervisionado, o sistema deverá ser capaz de distinguir, de forma eficiente, imagens normais e imagens que contenham falhas gráficas simuladas, como glitches, texturas corrompidas e distorções visuais.

Além disso, espera-se que este projeto contribua para demonstrar a viabilidade de integrar ferramentas de detecção automatizada de anomalias no pipeline de desenvolvimento de jogos digitais, auxiliando na garantia de qualidade visual e na redução de erros gráficos que possam comprometer a experiência do usuário, inclusive no aspecto de acessibilidade.

Por fim, os resultados deste trabalho poderão servir de base para pesquisas futuras, explorando o uso de inteligência artificial na automação de testes gráficos e na melhoria contínua da produção de jogos digitais.

\section{Cronograma}

O cronograma a seguir apresenta as atividades planejadas para o desenvolvimento deste trabalho, distribuídas ao longo dos meses previstos para sua execução.

\begin{table}[H]
\centering
\caption{Cronograma de atividades do trabalho}
\begin{tabular}{|l|c|c|c|c|c|c|}
\hline
\textbf{Atividades} & \textbf{Jul} & \textbf{Ago} & \textbf{Set} & \textbf{Out} & \textbf{Nov} & \textbf{Dez} \\ \hline
Revisão bibliográfica         & X & X &   &   &   &   \\ \hline
Coleta e preparação dos dados &   & X & X &   &   &   \\ \hline
Criação das anomalias         &   & X & X &   &   &   \\ \hline
Desenvolvimento do modelo     &   &   & X & X &   &   \\ \hline
Treinamento e testes          &   &   & X & X &   &   \\ \hline
Análise dos resultados        &   &   &   & X & X &   \\ \hline
Documentação e trabalho final &   &   &   & X & X & X \\ \hline
\end{tabular}
\label{tab:cronograma}
\end{table}
