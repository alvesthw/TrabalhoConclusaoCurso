\chapter{Trabalhos Correlatos}

Nesta seção são apresentados estudos que se relacionam com o tema da detecção de anomalias e da aplicação de técnicas de inteligência artificial em jogos digitais. O objetivo é situar a presente pesquisa no contexto das investigações já existentes, destacando as principais contribuições identificadas na literatura e apontando lacunas que justificam a proposta deste trabalho.

\section{Testes de Software Automatizados em Jogos Digitais}

O estudo de \citeonline{costa2024testes} aborda de forma abrangente o tema da automação de testes em jogos digitais, enfatizando a importância de mecanismos capazes de avaliar a qualidade, estabilidade e desempenho dos softwares de entretenimento. O autor ressalta que, com o aumento da complexidade dos ambientes virtuais e a multiplicidade de elementos gráficos e interativos, métodos tradicionais de verificação manual tornam-se inviáveis ou ineficientes. Dessa forma, técnicas automatizadas emergem como uma alternativa promissora para reduzir custos e tempo de desenvolvimento, além de minimizar a ocorrência de falhas não detectadas.

O trabalho apresenta um panorama sobre ferramentas e metodologias de automação, discutindo desde abordagens baseadas em scripts até o uso de algoritmos heurísticos para teste de comportamento. No entanto, observa-se que o estudo concentra-se principalmente em aspectos conceituais e técnicos da automação de processos, sem avançar para o campo do aprendizado de máquina. Em particular, o uso de redes neurais profundas para a detecção de anomalias visuais, como artefatos gráficos, erros de textura ou problemas de renderização, não é explorado. Essa lacuna evidencia a necessidade de investigações que apliquem técnicas de visão computacional no contexto de testes automatizados em jogos, o que motiva diretamente a proposta desta pesquisa.

\section{Redes Neurais Artificiais em Jogos Digitais}

O trabalho de \citeonline{nascimento2021} propõe o uso de redes neurais artificiais (RNAs) em jogos digitais com foco na adaptação dinâmica da dificuldade, especialmente em jogos de estratégia. Os autores demonstram que, ao coletar dados em tempo real sobre o desempenho e o comportamento do jogador, é possível ajustar parâmetros do jogo de forma automática, promovendo uma experiência personalizada e equilibrada. Essa aplicação evidencia o potencial das RNAs como ferramentas capazes de aprender e reagir a padrões complexos de interação.

Contudo, apesar de sua relevância para o campo da inteligência artificial aplicada a jogos, o estudo de \citeonline{nascimento2021} restringe-se à esfera da jogabilidade e do balanceamento dinâmico, sem abordar o uso de redes neurais voltadas à análise de aspectos visuais. Em particular, a pesquisa não contempla o emprego de redes neurais convolucionais (CNNs), que são amplamente reconhecidas por sua eficácia em tarefas de visão computacional, como classificação de imagens e detecção de anomalias visuais. Assim, o presente trabalho diferencia-se por explorar especificamente o potencial das CNNs na identificação automática de falhas gráficas em ambientes de jogos, contribuindo para a melhoria do controle de qualidade no processo de desenvolvimento.

\section{Síntese}

Os trabalhos analisados demonstram que tanto a automação de testes quanto a aplicação de técnicas de inteligência artificial representam tendências consolidadas na área de desenvolvimento de jogos digitais. Entretanto, observa-se que ainda existe uma lacuna significativa no que diz respeito à aplicação direta de modelos de aprendizado profundo para a detecção de anomalias visuais. 

Dessa forma, o presente TCC busca ampliar o escopo das pesquisas existentes ao propor uma abordagem baseada em redes neurais convolucionais (CNNs) voltada à análise automática de imagens de jogos. Ao integrar conceitos de visão computacional e aprendizado profundo ao contexto de testes de qualidade, este estudo pretende oferecer uma contribuição prática e inovadora, capaz de apoiar a indústria de jogos na identificação precoce de falhas gráficas e na otimização de seus processos de verificação e validação.

